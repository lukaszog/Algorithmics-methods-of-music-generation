\documentclass{article}
\usepackage{tikz}
\usetikzlibrary{matrix,chains,positioning,decorations.pathreplacing,arrows,calc}
\usepackage{polski}
\usepackage[utf8]{inputenc}
\tikzset{
	block/.style={
		draw,
		rectangle, 
		text width=3em, 
		text centered, 
		minimum height=8mm,     
		node distance=2.3em
	}, 
	line/.style={draw}
}

\begin{document}
	\begin{tikzpicture}[
	plain/.style={
		draw=none,
		fill=none,
	},
	net/.style={
		matrix of nodes,
		nodes={
			draw,
			circle,
			inner sep=10pt
		},
		nodes in empty cells,
		column sep=2cm,
		row sep=-9pt
	}, 
	>=latex
	]
	\matrix[net] (mat)
	{ 
		|[plain]| \parbox{1cm}{\centering Warstwa\\wejściowa} & |[plain]| \parbox{1cm}{\centering     Warstwa\\ukryta} & |[plain]| \parbox{1cm}{\centering Warstwa\\wyjściowa} \\
		& |[plain]| \\
		|[plain]| & \\
		& |[plain]| \\
		|[plain]| & |[plain]| \\
		& & \\
		|[plain]| & |[plain]| \\
		& |[plain]| \\
		|[plain]| & \\
		& |[plain]| \\
	};
	\foreach \ai [count=\mi ]in {2,4,...,10}
	\draw[<-] (mat-\ai-1) -- node[above] {Wejście \mi} +(-2cm,0);
	\foreach \ai in {2,4,...,10}
	{\foreach \aii in {3,6,9}
		\draw[->] (mat-\ai-1) -- (mat-\aii-2);
	}
	\foreach \ai in {3,6,9}
	\draw[->] (mat-\ai-2) -- (mat-6-3);
	%\draw[->] (mat-6-3) -- node[above] {Ouput} +(2cm,0);
	\path [line] node{błąd} -- (mat-1-1);
	\draw[->] (mat-6-3) -- ++(0pt,3cm) -| node[pos=0.15,above] {Wsteczna propagacja błędu} ( $ (mat-2-1)!0.5!(mat-2-2) $ );
	\end{tikzpicture}
	
\end{document}