\documentclass[11pt]{article}
\usepackage[textwidth=6.5in,textheight=8.5in]{geometry}
\usepackage[osf]{mathpazo}
\usepackage{textcomp}
\PassOptionsToPackage{urlcolor=black,colorlinks}{hyperref}
\RequirePackage{hyperref}
\usepackage{xcolor}
\newcommand{\myurl}[1]{\textcolor{blue}{\underline{\textcolor{black}{\url{#1}}}}}
\newcommand{\musixflxVersion}{0.83.3}
\begin{document}
\title{Installation of the CTAN MusiXTeX Fonts Distribution}
\author{Bob Tennent\\
\small\url{rdt@cs.queensu.ca}}
\date{\today}
\maketitle 
\thispagestyle{empty}

\section{Introduction}
Before trying to install from CTAN, check whether your TeX distribution
provides packages for MusiXTeX; this will be easier than doing it yourself.
But if your TeX distribution
doesn't have MusiXTeX (or doesn't have the most recent version), this distribution
of the MusiXTeX fonts is very easy to install, though
you may need to read the material on 
installation of (La)TeX files in the 
TeX FAQ\footnote{%
\myurl{http://www.tex.ac.uk/cgi-bin/texfaq2html}},
particularly
the pages on 
which tree to use\footnote{%
\myurl{http://www.tex.ac.uk/cgi-bin/texfaq2html?label=what-TDS}}
and installing files\footnote{%
\myurl{http://www.tex.ac.uk/cgi-bin/texfaq2html?label=inst-wlcf}}.

\section{Installing \texttt{musixtex-fonts.tds.zip}}

In this distribution of the MusiXTeX fonts, all of the files to be installed 
are in 
\begin{list}{}{}\item
\myurl{http://mirror.ctan.org/install/fonts/musixtex-fonts.tds.zip}
\end{list}
at CTAN. The archive is a zipped TEXMF
hierarchy; simply download and unzip this archive in the root folder/directory of whichever TEXMF tree
you decide is most appropriate, likely a ``local'' or ``personal'' one.
This should work with any TDS\footnote{%
\myurl{http://www.tex.ac.uk/cgi-bin/texfaq2html?label=tds}}
compliant TeX distribution, including MikTeX, TeXlive and teTeX.

After unzipping the archive, update the filename database as necessary,
for example, by executing \verb\texhash ~/texmf\ or 
clicking the button labelled ``Refresh FNDB" in the MikTeX settings program.

You now need to update various font-map files.  The details vary from one distribution
to another.  
On any current TeXLive-based system, or a teTeX v3.0 system, execute the command
\begin{list}{}{}\item
    \verb\updmap --enable MixedMap=musix.map\
\end{list}
if you've installed into a ``personal'' TEXMF tree, or 
\begin{list}{}{}\item
    \verb\updmap-sys --enable MixedMap=musix.map\
\end{list}
(as super-user) if you've installed to a ``local'' TEXMF tree.

On an older MiKTeX system, you may need to update the system file \verb\updmap.cfg\, using the shell command
\begin{list}{}{}\item
    \verb\initexmf --edit-config-file updmap\
\end{list}
adding the following line at the end if it isn't already there:
\begin{list}{}{}\item
    \verb\MixedMap musix.map\
\end{list}
Then generate revised font maps with the shell command
\begin{list}{}{}\item
    \verb\initexmf --mkmaps\
\end{list}

\section{Discussion}

Additional documentation, additional
add-on packages, and many examples of MusiXTeX typesetting may be found
at the Werner Icking Music Archive\footnote{%
\myurl{http://icking-music-archive.org}}.
Support for users of MusiXTeX and related software may be obtained via
the MusiXTeX mail list\footnote{%
\myurl{http://tug.org/mailman/listinfo/tex-music}}.
MusiXTeX fonts may be freely copied, duplicated and used in conformance to the
GNU General Public License (Version~2, 1991, see included file \verb\gpl.txt\).

\end{document}
